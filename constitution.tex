\documentclass[12pt]{constitution}
\usepackage{mathpazo}

\begin{document}


%% Commented out the Title and ToC pages
%\title{The Constitution of the United States of America}
%\author{Text from \url{http://www.usconstitution.net/const.txt}\\
%Section headings from \url{http://www.usconstitution.net%%/xconst.html}}
%\date{}
%\maketitle
%\setcounter{tocdepth}{0}
%\tableofcontents


\newpage

\chapter*{Data Science Student Society}

\vspace{-12pt} % too much of these are not good

\begin{center}
    \chancery{\LARGE{Preamble}}
\end{center}

We the students of UCSD, in order to promote a data-centric discipline in the spirit of Science, establish data literacy in the name of Education, provide revolutionary enlightenment in the new era of Big Data, and secure the knowledge of Data Science for our school and posterity, present the following document as the Constitution of the Data Science Student Society at UCSD.

\article{Name}
\section{}The formal name of the organization shall be "Data Science Student Society at UCSD."

\section{}Other recognized aliases include, but are not limited to, "DS3" and "DS3 at UCSD."


\article{Purpose}
\indent The Data Science Student Society at UCSD is an interdisciplinary academic organization designed to immerse the community in the diverse and growing facets of Data Science: Machine Learning, Computational Statistics, Data Mining, Visualization, Predictive Analytics, and any new emerging relevant fields of study. With practical hands-on data projects, a professional portfolio-building approach, and fun outreach activities, the Data Science Student Society at UCSD strives to enrich the academic life of the student community by strengthening them for success in their current and future pursuits of Data Science related fields.

\article{Non-Profit Status}
\section{} Data Science Student Society at UCSD is a non-profit student organization.

\article{Membership}
\section{} General membership for the Data Science Student Society at UCSD is open to all registered UCSD students, either undergraduate or graduate.
\section{} Principal membership is only open to registered undergraduate UCSD students.
\section{} Team membership for data projects may be determined and organized by principal members or an officer committee.
\section{} There are no registration fees due for general membership.

\article{Meetings}
\section{Organizer}Principal members or an officer committee should hold weekly meetings.
\section{General}The frequency of General Body Meetings is determined by principal members or an officer committee.

\article{Advisors}
\section{Community} The community advisor can be chosen from within or outside the UCSD community by principal members or an officer committee.
\section{Academic} The academic faculty or staff advisor from an interdisciplinary department or program associated with Data Science can be chosen by principal members or an officer committee.

\article{Officership}

\section{Prerequisite}Only an undergraduate UCSD student may become a principal member or an officer for the Data Science Student Society at UCSD.

\section{Qualification} The candidate demonstrates Data Science knowledge from experience or presents at least one data science-related project.

\section{Concurrency} No member may hold more than one officer position during their time in office. 

\section{Voting} Only a registered undergraduate UCSD student may vote in elections for the selection of officers and can have up to one vote for each officer candidate, but registered undergraduate students who are running for officer candidacy in an election may not vote for the selection of officers in that election.

\section{Season} Election period and voting season for officers should begin every Spring quarter, preferably within the first 3 weeks.

\section{Term Length} Officers are usually expected to hold office for one school year.

\section{Positions} {

The following is a summary description of the elected and formally recognized, yet not fully comprehensive, officer positions along with their associated baseline responsibilities:

	\begin{itemize}
  		\item President(s)  {
        
       \par\indent
        	Oversees vitality of the organization. Initiates important decisions, with the advisement and teamwork of other fellow officers, as to the organization's programs, events, fundraising, marketing, recruitment, and all other tasks and duties required to keep the organization functional and effective.
        }
        \item Vice President of Internal {
        
        \par\indent
        	Takes minutes of executive committee meetings, checks in with teams, serves as the Human Resource arm of the organization, plans internal bonding events. Guides discussion about officer positions that need to be amended, added, and/or removed from the Constitution.
        }
  		\item Vice President of Finance {
        
        \par\indent
        	Manages fundraisers and keeps the organization's books debt free. In charge of providing a quarterly budget to A.S. finance committee.
        }
  		\item Vice President of Business Relations {
        
        \par\indent
        	Researches and lobbies companies for projects that organization members can work on. Oversees general relationship between the organization with companies and sponsorships.
        }
        
  		\item Vice President of Alumni Relations {
        
        \par\indent
        	Plans alumni events and maintains a network between the organization and its alumni.
        }
        
        \item Director of Marketing {
        
        \par\indent
        	Leads marketing campaigns for the organization and its activities on campus and through social media.
        }
        
        \item Director of Education {
        
        \par\indent
        	Plans educational, teaching, and workshop sessions. Oversees lesson plans for internal users. Researches methods used by professors, lecturers, and industry professionals in order to learn Data Science effectively.
        }
        
	\end{itemize}
}

\section{Principalities}
There shall be four or more principal members up to a maximum of eight who are chosen by a committee comprised of all current officers.


\section{Tiebreaking}
Ties in an election for an officer will be resolved by using the most recent student vote as the casting vote. Ties in decisions by officer committees will be resolved by using one randomly chosen officer as the person who makes the casting vote.

\article{Amendment}
\section{} The principal members and officer committee should review the Constitution periodically and implement revisions after thorough discussion, equal representation, and formal process.

\article{Motto}

\begin{center}
    "Data is a precious thing and will last longer than the systems themselves."\\
    --- Timothy J Berners-Lee
\end{center}


\end{document}